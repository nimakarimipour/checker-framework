\htmlhr
\chapterAndLabel{Returns Receiver Checker}{returns-receiver-checker}

The Returns Receiver Checker enables documenting and checking that a method
returns its receiver (i.e., the \<this> parameter).

There are two ways to run the Returns Receiver Checker.
\begin{itemize}
\item
Typically, it is automatically run by another type checker.
\item
Alternately, you can run just the Returns Receiver Checker, by
supplying the following command-line options to javac:
\code{-processor org.checkerframework.common.returnsrcvr.ReturnsRcvrChecker}
\end{itemize}


\sectionAndLabel{Annotations}{returns-receiver-checker-annotations}

With the Returns Receiver Checker, the type annotation
\refqualclass{common/returnsrcvr/qual}{This} on the return type of a method
indicates that the method should return its receiver.  The
\refqualclass{common/returnsrcvr/qual}{This} annotation should only be written
on the return type of a method, or (in rare cases) as part of a downcast.  When
using certain frameworks for the builder pattern,
\refqualclass{common/returnsrcvr/qual}{This} annotations will be injected
automatically; see Section~\ref{returns-receiver-checker-framework-support}. 


\sectionAndLabel{Warnings}{returns-receiver-checker-warnings}

The \refqualclass{common/returnsrcvr/qual}{This} annotation is a polymorphic
qualifier (see Section~\ref{qualifier-polymorphism}).  When applied to a method
with an \refqualclass{common/returnsrcvr/qual}{This} annotation on its return
type, the Returns Receiver Checker injects an
\refqualclass{common/returnsrcvr/qual}{This} annotation on the method's receiver
parameter.  The presence of the polymorphic
\refqualclass{common/returnsrcvr/qual}{This} qualifier on both the method's
return and receiver type force their types to be equal.  Hence, Checker
Framework's standard checking of type compatibility will discover cases where a
method is not properly returning its receiver.  When a method with an
\refqualclass{common/returnsrcvr/qual}{This} return type returns some value
other than its receiver, the issue is reported as a
\code{[return.type.incompatible]} warning.  Similarly, if a method with an
\refqualclass{common/returnsrcvr/qual}{This} return type is overridden by a
method lacking \refqualclass{common/returnsrcvr/qual}{This} on its return type,
the checker reports an \code{[override.return.invalid]} warning.

Additionally, an \code{[invalid.this.location]} warning will be emitted if an
\refqualclass{common/returnsrcvr/qual}{This} annotation is written in any
location other than on the return type of a method or in a downcast.


\sectionAndLabel{Framework Support}{returns-receiver-checker-framework-support}

\begin{figure}
    \begin{Verbatim}
    @AutoValue
    abstract class Animal {
        abstract String name();
        abstract int numberOfLegs();
        static Builder builder() {
            return new AutoValue_Animal.Builder();
        }
        @AutoValue.Builder
        abstract static class Builder {
            abstract Builder setName(String value);
            abstract Builder setNumberOfLegs(int value);
            abstract Animal build();
        }
    }            
    \end{Verbatim}
    \caption{A code example using AutoValue's Builder support.}
    \label{fig-autovalue-builder}
\end{figure}

For programs using builder generation support from
\href{https://github.com/google/auto/tree/master/value}{AutoValue} or
\href{https://projectlombok.org/}{Project Lombok}, the Returns Receiver Checker
will automatically inject \refqualclass{common/returnsrcvr/qual}{This}
annotations on the return types of the fluent API methods of generated builders.
Consider the code using AutoValue in Figure~\ref{fig-autovalue-builder}.  For
this code, AutoValue automatically generates a class
\<AutoValue\_Animal.Builder> that extends \<Animal.Builder> and implements its
abstract methods.  The Returns Receiver Checker automatically adds a
\refqualclass{common/returnsrcvr/qual}{This} annotation to the return type of
\<Animal.Builder.setName> and \<Animal.Builder.setNumberOfLegs>, and also to the
return types of those methods in the generated subclass.  The Returns Receiver
Checker still runs to ensure the implementation of the generated subclass is
consistent with the injected annotations.  Project Lombok Builders are supported
in a similar fashion.

To disable the checker's built-in framework support, use the
\<-AdisableFrameworkSupports> command-line option, passing a list of frameworks
whose support should be disabled.  For example, to disable both AutoValue and
Lombok support, pass \<-AdisableFrameworkSupports=AutoValue,Lombok>.

