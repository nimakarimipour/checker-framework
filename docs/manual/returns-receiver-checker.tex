\htmlhr
\chapterAndLabel{Returns Receiver Checker}{returns-receiver-checker}

The Returns Receiver Checker enables documenting and checking that a method
returns its receiver (i.e., the \<this> parameter).

There are two ways to run the Returns Receiver Checker.
\begin{itemize}
\item
Typically, it is automatically run by another checker.
\item
Alternately, you can run just the Returns Receiver Checker, by
supplying the following command-line options to javac:
\code{-processor org.checkerframework.common.returnsrcvr.ReturnsRcvrChecker}
\end{itemize}


\sectionAndLabel{Annotations}{returns-receiver-checker-annotations}

The type annotation \refqualclass{common/returnsrcvr/qual}{This} on the return
type of a method indicates that the method returns its receiver. As an example,
Figure~\ref{fig-autovalue-builder-generated} shows a \<Builder> class in which
\refqualclass{common/returnsrcvr/qual}{This} is used to specify that the
\<setName> and \<setNumberOfLegs> methods return their receiver.  An
\refqualclass{common/returnsrcvr/qual}{This} annotation can only be written on
the return type of a method or in a downcast; if written elsewhere, the checker
will emit an \code{[invalid.this.location]} warning.

The qualifier hierarchy and type-checking approach for the Returns Receiver
Checker are both somewhat unusual.  As is standard, the Returns Receiver
Checker has a top qualifier,
\refqualclass{common/returnsrcvr/qual}{MaybeThis}, and a bottom qualifier,
\refqualclass{common/returnsrcvr/qual}{BottomThis}.  However, \refqualclass{common/returnsrcvr/qual}{This}
is \emph{not} a subtype of \refqualclass{common/returnsrcvr/qual}{MaybeThis},
but instead a polymorphic qualifier.  While polymorphic qualifiers are typically
used to allow a method to have multiple qualified type signatures (see
Section~\ref{qualifier-polymorphism}), here polymorphism is used to equate a
method's return and receiver type qualifier.  When the checker observes an
\refqualclass{common/returnsrcvr/qual}{This} annotation on a method return type,
it proceeds to automatically add \refqualclass{common/returnsrcvr/qual}{This} to
the method's receiver type. The presence of
\refqualclass{common/returnsrcvr/qual}{This} on both the method's return and
receiver type forces their type qualifiers to be \emph{equal}.  Hence, the
method will only pass the type checker if it returns its receiver argument,
achieving the desired checking.

\sectionAndLabel{AutoValue and Lombok Support}{returns-receiver-checker-autovalue-lombok-support}

\begin{figure}
    \begin{Verbatim}
    @AutoValue
    abstract class Animal {
        abstract String name();
        abstract int numberOfLegs();
        static Builder builder() {
            return new AutoValue_Animal.Builder();
        }
        @AutoValue.Builder
        abstract static class Builder {
            abstract @This Builder setName(String value);
            abstract @This Builder setNumberOfLegs(int value);
            abstract Animal build();
        }
    }
    \end{Verbatim}
    \caption{An illustrative use of the \<@AutoValue.Builder> annotation.
    Given this code, AutoValue automatically generates a concrete subclass of
    \<Animal.Builder> (see Figure~\ref{fig-autovalue-builder-generated}).  The <@This> annotations on the setters in
    \<Builder> are automatically added by the Returns Receiver Checker.}
    \label{fig-autovalue-builder}
\end{figure}

\begin{figure}
    \begin{Verbatim}
    class AutoValue_Animal {
        static final class Builder extends Animal.Builder {
            private String name;
            private Integer numberOfLegs;
            @This Animal.Builder setName(String name) {
              this.name = name;
              return this;
            }
            @This Animal.Builder setNumberOfLegs(int numberOfLegs) {
              this.numberOfLegs = numberOfLegs;
              return this;
            }
            @Override
            Animal build() {
              return new AutoValue_Animal(
                  this.name,
                  this.numberOfLegs);
            }
          }
    }
    \end{Verbatim}
    \caption{Code generated by AutoValue for the example of
    Figure~\ref{fig-autovalue-builder}, including the \<@This> annotations added
    by the Returns Receiver Checker.}
    \label{fig-autovalue-builder-generated}
\end{figure}

The \href{https://github.com/google/auto/tree/master/value}{AutoValue} and
\href{https://projectlombok.org/}{Lombok} projects both support automatic
generation of builder classes, which enable flexible object construction.
Figure~\ref{fig-autovalue-builder} shows code using AutoValue to specify the
methods in a builder (using the \<@AutoValue.Builder> annotation), and
Figure~\ref{fig-autovalue-builder-generated} shows the builder implementation
that AutoValue generates for this example.  For code using these two frameworks,
the Returns Receiver Checker automatically adds \<@This> annotations to setter
methods in builder classes.  For example, all the \<@This> annotations in
Figures~\ref{fig-autovalue-builder} and~\ref{fig-autovalue-builder-generated}
are automatically added by the checker.

To disable the checker's built-in framework support, use the
\<-AdisableFrameworkSupport> command-line option, passing a list of frameworks
whose support should be disabled.  For example, to disable both AutoValue and
Lombok support, pass \<-AdisableFrameworkSupport=AutoValue,Lombok>.
